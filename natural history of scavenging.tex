\documentclass[a4paper,12pt]{article}
\usepackage{ecography}
\usepackage{lmodern}


\title{The natural history of scavenging in vertebrates}
\running{Scavenging in vertebrates}

\author{Adam Kane, \and Kevin Healy, \and Thomas Guillerme \and Graeme Ruxton, \and \& Andrew Jackson.}

\affiliations{
\item A. Kane (\url{adam.kane@ucc.ie}), University College Cork, Cooperage Building\, School of Biological 
Earth and Environmental Sciences\,Cork, Ireland.
\item K. Healy, T. Guillerme and A. Jackson, Trinity College Dublin, Department of Zoology;
School of Natural Sciences, Dublin Ireland.
\item G. Ruxton, School of Biology, Sir Harold Mitchell Building, Greenside Place, St Andrews, KY16 9TH,
United Kingdom
}

\nwords{9999}


\begin{document}


\maketitle


\begin{abstract}
  Scavengers existed in the past and they exist now. 
  Often under appreciated. 
  Three main habitat types considered: land, air and sea. 
  Different drivers in these areas. 
  Review looks at these 
\end{abstract}


\newpage


\section*{Introduction}
Scavenging is a widespread behaviour amonng vertebrates where most if not all carnivores act as facultative scavengers. 
It is recognised that scavengers have an important role in keeping energy flows at a higher trophic level in food webs than decomposers because they consume relatively more carrion \citep{devault2003scavenging}. 
Scavengers also provide useful ecosystem services by acting as barriers to the spread of disease by quickly consuming rotting carcasses which have often died from illness \citep{ogada2012dropping}.
%(Since we are intrested in scavanging in th paleo record ecosystem services might not be that relavent, although modulators of disease is still relavent) 
Despite this, scavengers are a seriously understudied group \citep{sekercioglu2006increasing,selva2007nested,wilson2011scavenging}.  
\cite{devault2003scavenging} propose that this is due to both human disgust at carrion itself and the difficulty in determining if an ingested prey item was killed or scavenged. 
The latter point means that studying the natural history of this behaviour is particularly problematic. Indeed, even data on the proportion of carrion in the diet of extant species are sorely lacking \citep{benbow2015introduction}. The limitations in studying extant scavenging behviour is much larger in extinct species and systems with the obvious lack of observational data available. This means indirect observations in the fossil record and other approaches such as energetics must be used to infer these behaviours.  
%
\\ One avenue to infer scavenging from palaeontological data can be achieved by determining if a prey item was simply too big for the carnivore to have tackled in cases where tooth marks are found \citep{pobiner2008paleoecological}. 
Comparative analysis can also allow us to ascertain which morphologies and physiologies are likely to have been found in scavenging species in the past \citep{ruxton2004obligate}.
The development of indirect measures of scavenging in paleontology can in turn be applied to current scavenging systems that also suffer from a lack of observational data. In this review we collate methods (could this be another way of structuring it, just an idea) and research form palaeontology relating to scavenging behaviour and show that ignore this literature would be a missed opportunity for understanding extant scavenging.


\\ Our review is divided up into three sections, namely the land, air and sea. 
These are then subdivided into three geological eras the Cenozoic, Mesozoic and Palaeozoic. 
Each of these environments has a distinct phyiscal character that affects how a species forages for food which has obvious relevance for an animal searching for carrion.  
However, there is some commonality to a scavenger's environment and the problems in finding food that one would encounter. 
Notably, the resource environment of a scavenger is a patchily distributed one, because it is difficult to predict both when and where a carcass is produced. 
As a result of this, any animal existing as a scavenger must maximise its detection capabilities and minimise its locomotory costs \citep{ruxton2004obligate}.
Exploring which groups are likely to have moved towards these traits and thus existed as sccavengers over palaentological time forms the basis of this work.
%remember the journal is an ecological one at heart so I would always keep in mind about making it useful for ecologists over paelo people. In particular from the website "ECOGRAPHY publishes papers focused on broad spatial and temporal patterns, particularly studies of population and community ecology, macroecology, biogeography, and ecological conservation. Studies in ecological genetics and historical ecology are welcomed in the context of explaining contemporary ecological patterns".

\section{Terrestrial Scavengers}
As noted above the ease with which natural selection pushes an animal towards the optima of cheap locomotion and large detection capabilities is dependent on the environment the species is in. 
Land-based scavengers can be thought of as existing in a 2-dimensional plane while foraging for carrion directly. 
The range at which they can detect carcasses is thus defined by the radius of their sensory organs, usually the visual and olfactory senses. 

\subsection*{Cenozoic}
Among terrestrial African carnivores, hyenas, jackals, lions and leopards all take sizable proportions of carrion in their diet.
In the case of the spotted hyena (\textit{Crocuta crocuta}) it can be as high as 99\% \citep{benbow2015introduction}. 
Yet, no contemporary terrestrial vertebrate exists as an obligate scavenger. 
The selective pressures that push mammals and reptiles towards scavenging do not seem to undermine their ability to hunt, perhaps explaining the absence of obligate scavengers in these groups \citep{ruxton2004obligate}.
\\There is a long running debate on the tendency of human ancestors to scavenge. 
Some recent studies have found "that passive scavenging from abandoned larger felid kills could have been a high-yield, though potentially dangerous, foraging strategy for early hominins even without considering within-bone nutrients" \citep{pobiner2015new}. 

Osteophagy is known across a range of terrestrial carnivores.
Some fat-rich mammalian bones have an energy density (6.7 kJ/g) comparable with that of muscle tissue, making skeletal remains an enticing resource \citep{brown1989study}. 
Hyenas have a bite force capable of breaking open bones and early hominins had the ability to craft tools for this purpose \citep{hone2010feeding,ARCM:ARCM12084}. 
In light of this, the skeletal remains of carrion may act as trove of food to carnivores who can access it.  

\cite{ruxton2004obligate} in a theoretical study suggested that "a 1 tonne mammal or reptile, in an ecosystem yielding carrion at densities similar to the current Serengeti, could have met its energy requirements if it could detect carrion over a distance of the order of 400–500 m."

\subsection*{Mesozoic}
In a recent publication a modelling approach found that theropods between 27 kg and 1044 kg would have gained a significant energetic advantage over individuals at both the small and large extremes of theropod body mass through their scavenging efficiency. 
This humped pattern is the result of the disparity between the scaling of energetic cost which scales according to an exponent of 0.91 and energy input that scales according to a cubic polynomial that initially scales according to an exponent of 1.07 but plateaus after 1000kg. 
The polynomial behaviour of energy input is itself the result of the limitations imposed by gut capacity and the overall availability of carcasses after competition.

As we discussed for the case of Cenozoic carnivores, osteophagy could be extremely beneficial to a scavenger. 
In Mesozoic systems some extremely large theropod dinosaurs had a morphology which suggests an ability to process bone e.g. the robust skull and dentition of \textit{Tyrannosaurus rex}. 
There is direct evidence that \textit{T.rex} did this in the form of distinctive wear marks on its tooth apices \citep{farlow1994wear,schubert2005wear} and the presence of bone fragments in its coprolites \citep{chin1998king}. 
The animal also had an enormous bite force, with one estimate putting it at 57000 Newtons \citep{bates2012estimating}. 
This is noted as being powerful enough to break open skeletal material \citep{rayfield2001cranial}. 
Osteophagy may have been even more viable during this era because the body mass distribution of herbviores tended to be skewed towards larger sizes \citep{10.1371/journal.pone.0051925}. 
When we couple this with the fact that skeletal mass scales greater than linearly with body mass \citep{prange1979scaling} there would have been a lot of bones to consume in the environment provided an animal had the biology to process it. 


\textit{Allosaurus} tooth marks on a hadrosaur in the Late Jurassic \citep{chure1997one}. 
Late Triassic scavenging on a prosauropod by basal carnivorous archosaurs \citep{hungerbuhler1998taphonomy}.

\subsection*{Palaeozoic}
Synapsids 
Sprawling gait



\section{Aerial Scavengers}
Species capable of flight have effectively added an extra spatial dimension, i.e. the vertical component, to their sensory environment.
This allows them to look down on a landscape where they are unencumbered by obstacles that would obstruct the view of a terrestrial scavenger.
Moreover, flight is a cheaper means of locomotion than running \citep{tucker1975energetic}. 
Thus, it appear that would-be scavengers have a distinct advantage by taking flight. 

\subsection*{Cenozoic}
Birds are the dominant vertebrate fliers today and include the best known scavengers on Earth, the vultures. 
These birds consist of two convergent groups, old world and new world vultures and represent the only example of obligate vertebrate scavengers. 
They have a suite of adaptations that allow them to flourish as obligate scavengers.
Vultures extend the energetic advantage of flight further by engaging in soaring instead of flapping flight, which is even cheaper \citep{hedenstrom1993migration}.
Their efficiency is illustrated by cases of predators like bears and wolves benefiting by taking more carrion in their diet in areas bereft of vultures through competitive release \citep{devault2003scavenging}. 
In flight, birds possess a huge advantage over any terrestrial obligate scavenger. 
Flight affords them the ability to range over a much larger area and detect carrion from an elevated vantage point.
\cite{pennycuick1972soaring} conservatively estimated that a Gyps vulture could identify activity at a carcass 4 km away. 

Many other bird species take a significant amount of carrion in their diet notably the eagles, storks and corvids. 
Although, none of them are obligate scavengers. 
By contrast with terrestrial species, the traits that render a bird adept at scavenging do undermine its ability to function as an effective predator \citep{devault2003scavenging}. 

\subsection*{Mesozoic}
In ancient ecosystems, the volant pterosaurs have also been postulated as existing in a vulture-like niche \citep{witton2008reappraisal}. 
Certain clades of these animals could reach enormous sizes (e.g. Azhdarchids) and look to have engaged in soaring flight. 
However, the inflexibility of their necks and straight, rather than hooked jaw morphology argues against their existing as obligate scavengers \citep{witton2008reappraisal}. 
As yet, no one has ever attempted an energetics approach for this group which is likely due to the many uncertainties over their biology \citep{witton2010size}. 

\subsection*{Palaeozoic}
The absence of flying vertebrates in the Palaeozoic may have permitted terrestrial forms to take in a higher proportion of carrion in their diet. 




\section{Aquatic Scavengers}
The existence of an obligate scavenger in a marine setting also remains hypothetical \citep{britton1994marine,smith2003ecology,ruxton2004energetic,ruxton2005searching}. 
Carrion in this environment is produced by dead flesh and marine organisms when their carcasses descend to the sea floor. 
This low-light environment means animals detect resources through chemo- and mechanoreception \citep{ruxton2004energetic}. 
Detection distances are far lower than they would be in the air (< 100 m) as a result. 
However, water is a medium that is conducive to low-cost movement \citep{tucker1975energetic} and so may be able to support a small obligate scavenging fish \citep{ruxton2004energetic,ruxton2005searching}. 
Although, for the time being this remains conjectural.

\subsection*{Cenozoic}
A likely instance of scavenging between a 4-million-year-old white shark (\textit{Carcharodon}) and mysticete whale from Peru \citep{ehret2009caught}.
Bite marks on early Holocene Tursiops truncatus fossils from the North Sea indicate scavenging by rays (Chondrichthyes, Rajidae) \citep{van2009bite}. 
Possible scavenging on a juvenile fur seal from the Late Neogene \citep{boessenecker2011mammalian}. 

\subsection*{Mesozoic}
Evidence of scavenging in a Cretaceous shark species \textit{Squalicorax} whereby remains of a mosasaur and a hadrosaur were discovered with embedded shark teeth \citep{schwimmer1997scavenging}. 
\subsection*{Palaeozoic}
Evolution of sharks, known scavengers. 
Evidence of vertebrate scavenging dates back to the early Permian approximately 300 MYA \citep{reisz2006articulated}.




\section*{Results}




\section*{Discussion}



\section*{Acknowledgments}

A lot of people are to thank here.


\newpage


\bibliography{bibfile}



\end{document}
